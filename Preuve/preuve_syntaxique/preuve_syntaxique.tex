\documentclass[12pt]{article}

\usepackage[utf8]{inputenc}
\usepackage[T1]{fontenc}
\usepackage[french]{babel}
\usepackage{amsmath, amssymb}
\usepackage{stmaryrd}
\usepackage{comment}
\usepackage{ulem}
\usepackage{ebproof}


\title{%
  Coalescing - Expansion d'opérateur \\
  \normalsize Preuve en utilisant la déduction naturelle}
\author{Raphaël Le Bihan}

\newcommand{\defeq}{\ensuremath{\; \hat{=} \;}}
\newcommand{\FOL}{\ensuremath{\textup{\tiny{}FOL}}}
\newcommand{\false}{\textup{ff}}
\newcommand{\true}{\textup{tt}}
\newcommand{\M}{\ensuremath{\mathcal{M}}}
\newcommand{\I}{\ensuremath{\mathcal{I}}}
\newcommand{\dom}{\ensuremath{\textup{ dom }}}

\usepackage{xcolor}
\newcommand{\raph}[1]{\textcolor{red}{#1}}

\definecolor{myblue}{rgb}{0,0.5,0.8}
\newcommand{\mpar}[1]{\marginpar{\color{red}\footnotesize\raggedright#1}}
\newcommand{\bpar}[1]{\marginpar{\color{myblue}\footnotesize\raggedright#1}}
\addtolength{\marginparwidth}{1cm}
\long\def\stephan#1{{\color{myblue} #1}}
\newcommand{\TRUE}{\textsc{true}}
\newcommand{\FALSE}{\textsc{false}}

\newtheorem{prop}{Propriété}
\newtheorem{lem}{Lemme}

\begin{document}

\maketitle

On étudie dans ce document le remplacement d'un opérateur par sa définition dans une formule FOML.
On cherche à montrer en particulier que pour un opérateur $d(\vec{x_i}) \triangleq e_d$ et une formule FOML $\phi$, si $\phi$ est valide une fois abstraite en FOL, alors la formule $\widetilde{\phi}$ obtenue en remplaçant chaque occurence de $d(\vec{e_i})$ par $e_d(\vec{e_i}/\vec{x_i})$ est également valide une fois abstraite en FOL.

On cherche alors à établir une correspondance sémantique entre les abstractions FOL de $\phi$ et $\widetilde{\phi}$.
\[
  \phi \to \phi_\FOL
\]
\[
  \phi \to \widetilde{\phi} \to \widetilde{\phi}_\FOL
\]

\raph{%
  (J'effacerai ce schéma ou j'en ferai un mieux, mais pour l'instant il permet de comprendre quelles formules qu'on manipule.)}

\paragraph{Définition} Remplacement d'un opérateur défini\\
\bpar{On suppose que $d$ n'apparaît pas dans $e_d$?}
Soit $d(\vec{x_i}) \triangleq e_d$ un opérateur défini, et $\phi$ une formule FOML pouvant contenir des occurences de $d$.
On définit récursivement $\widetilde{\phi}$ la formule obtenue en remplaçant $d$ par sa définition dans $\phi$ par :
\begin{align*}
  \widetilde{x} &\defeq x \\
  \widetilde{v} &\defeq v \\
  \widetilde{op(\vec{e_i})} &\defeq op(\vec{\widetilde{e_i}}) 
  \qquad\text{\stephan{si $op$ différent de $d$}}\\
  \widetilde{e_1 = e_2} &\defeq \widetilde{e_1} = \widetilde{e_2} \\
  \widetilde{\textup{FALSE}} &\defeq \textup{FALSE} \\
  \widetilde{e_1 \Rightarrow e_2} &\defeq \widetilde{e_1} \Rightarrow \widetilde{e_2} \\
  \widetilde{\forall x : e } &\defeq \forall x : \widetilde{e} \\
  \widetilde{\nabla e} &\defeq \nabla \widetilde{e} \\
  \widetilde{d(\vec{e_i})} &\defeq e_d(\vec{\widetilde{e_i}}/\vec{x_i})
\end{align*}

\raph{%
  Cette définition correspond bien à ce qu'on veut faire.\\
  Remarque 1 : Ici on n'a pas parlé des opérateurs définis autres que d.\\
  Remarque 2 : On ne remplace pas «qu'une fois» $d$, mais récursivement.
  Par ex $d(d(e))$ donne $e_d([e_d(e/x)]/x)$ et pas $e_d([d(e)]/x)$.}

\stephan{%
  $e_d(\vec{\widetilde{e_i}}/\vec{x_i})$ plutôt que $\widetilde{e_d}(\vec{\widetilde{e_i}}/\vec{x_i})$ suffit car $d$ ne peut apparaître dans $e_d$ et sinon ça ne terminerait pas. S'il y a plusieurs opérateurs définis (toujours sans récursion) il faudra les éliminer l'un après l'autre.}

\begin{prop}
  \label{prop-sem}
Soit $d(\vec{x_i}) \triangleq e_d$ un opérateur défini, et $\phi$ une formule FOML pouvant contenir des occurences de $d$.
Si $\vDash_\FOL \phi_\FOL$ alors $\vDash_\FOL \widetilde{\phi}_\FOL$.
\end{prop}

\paragraph{Preuve de la propriété \ref{prop-sem}}
Par complétude et correction de la déduction naturelle, il suffit de montrer la propriété \ref{prop-syn}.

\begin{prop}
  \label{prop-syn}
  Pour toute formule FOML $\phi$ et tout ensemble de formules FOML $\Gamma$,
  si $\Gamma_\FOL \vdash \phi_\FOL$, alors $\widetilde{\Gamma}_\FOL \vdash \widetilde{\phi}_\FOL$.
\end{prop}

\paragraph{Preuve de la propriété \ref{prop-syn}}
On fait une preuve par récurrence sur la dérivation de $\Gamma_\FOL \vdash \phi_\FOL$en considérant la dernière règle étudiée.

La logique FOL étudiée ne contient pas les symboles $\land, \lor, \exists, \neg$.
On étudiera seulement les règles suivantes : ax, RAA, $\Rightarrow_I$, $\Rightarrow_E$, $\forall_I$, $\forall_E$, $\bot_E$.

On montre les cas RAA, $\Rightarrow_I$, $\Rightarrow_E$, $\bot_E$ en appliquant les définitions et l'hypothèse de récurrence.
Le cas ax est trivial.
On considère maintenant les cas $\forall_I$ et $\forall_E$.

\begin{itemize}
\item
  Cas $\forall_I$ :\\
  On suppose qu'il existe une preuve de
  \begin{prooftree}
    \hypo{ \Gamma_\FOL \vdash \psi }
    \infer1[$\forall_I$]
    { \Gamma_\FOL \vdash \phi_\FOL = \forall x : \psi }
  \end{prooftree}, où $x \notin fv(\Gamma_\FOL)$.
  
  Par analyse de cas sur la définition de $\phi_\FOL$ on trouve que $\phi = \forall x : \phi_1$ avec ${\phi_1}_\FOL = \psi$ :
  \begin{prooftree}
    \hypo{ \Gamma_\FOL \vdash {\phi_1}_\FOL }
    \infer1[$\forall_I$]
    { \Gamma_\FOL \vdash \forall x : {\phi_1}_\FOL }
  \end{prooftree}

  On peut alors appliquer l'hypothèse de récurrence, il existe donc une dérivation de $\widetilde{\Gamma}_\FOL \vdash \widetilde{\phi_1}_\FOL$.
  
  On peut montrer que si $x \notin fv(e_\FOL)$, alors $x \notin fv(\widetilde{e}_\FOL)$ par récurrence sur $e$. On en déduit que $x \notin fv(\widetilde{\Gamma}_\FOL)$.

  On applique donc la règle $\forall_I$ pour obtenir la dérivation suivante :
  \begin{prooftree}
    \hypo{ \widetilde{\Gamma}_\FOL \vdash \widetilde{\phi_1}_\FOL }
    \infer1[$\forall_I$]
    { \widetilde{\Gamma}_\FOL \vdash \forall x : \widetilde{\phi_1}_\FOL } 
  \end{prooftree}
  où $\forall x : \widetilde{\phi_1}_\FOL = \widetilde{(\forall x : \phi_1)}_\FOL = \widetilde{\phi}_\FOL$.


  \vspace{1cm}
\item
  Cas $\forall_E$ : \raph{C'est ce cas qui est le plus dur} \\
  On suppose qu'il existe une preuve de
  \begin{prooftree}
    \hypo{ \Gamma_\FOL \vdash \forall x : \psi }
    \infer1[$\forall_E$]
    { \Gamma_\FOL \vdash \phi_\FOL = \psi\{x \mapsto t\} }
  \end{prooftree}.

  \raph{%
    On aimerait pouvoir appliquer l'hypothèse de récurrence sur $\Gamma_\FOL \vdash \forall x : \psi$. \\
    Pour ça il suffirait de montrer qu'il existe $\theta$ tel que $\theta_\FOL = \psi$.
    On veut montrer le lemme \ref{lemme-theta}. \\
    Le problème est que ce n'est pas toujours le cas, voir le contre exemple proposé à la fin du document.}
\end{itemize}

\begin{lem}
  \label{lemme-theta}
  Soit $\phi$ une formule FOML, $\psi, t$ des formules FOL telles que $\phi_\FOL = \psi\{x \mapsto t\}$.
  Alors il existe $\theta, \zeta$ des formules FOML telles que :
  \begin{enumerate}
  \item
    $\theta_\FOL = \psi$
  \item
    $\zeta_\FOL = t$ ou $x \notin fv(\psi)$
  \item
    $\zeta$ est rigide, ou $x$ n'apparait qu'à des positions Leibniz dans $\theta$
  \end{enumerate}
\end{lem}

\paragraph{Preuve du lemme \ref{lemme-theta}}

\raph{On n'arrivera pas à montrer tous les cas.}\\
On fait une preuve par récurrence sur $\psi$ et on montre la propriété avec $\phi_\FOL^z$ pour tout $z$.

On arrive effectivement à montrer les cas où $\psi = v, y (\neq x), x, \textup{FALSE}$.
La preuve fonctionne aussi pour les cas où $\psi = op(\vec{\psi_i}), \psi_1 \Rightarrow \psi_2, \psi_1 = \psi_2, \forall x : \psi_1$
La preuve échoue pour le cas \(\psi = \framebox{$d_{\vec{\epsilon}}$}(\vec{\psi_i}) \).

\paragraph{Contre-exemple au lemme \ref{lemme-theta}}

On considère un opérateur défini $d$ prenant un argument dont la position n'est pas Leibniz. \\
On pose :
\[ \phi = d(v) \]
Alors \(\phi_\FOL = \framebox{$d_v$}(v) = (\framebox{$d_v$}(x))\{x \mapsto v\} \).
On pose :
\[ \psi = \framebox{$d_v$}(x) \]
\[ t = v \]
On peut montrer qu'il n'existe pas de formule FOML $\theta$ telle que $\theta_\FOL = \psi$.
\end{document}
