\documentclass[12pt]{article}

\usepackage[utf8]{inputenc}
\usepackage[T1]{fontenc}
\usepackage[french]{babel}
\usepackage{amsmath, amssymb}
\usepackage{stmaryrd}
\usepackage{comment}
\usepackage{ulem}
\usepackage{ebproof}


\title{%
  Coalescing - Expansion d'opérateur \\
  \normalsize Preuve en utilisant la déduction naturelle}
\author{Raphaël Le Bihan}

\newcommand{\defeq}{\ensuremath{\; \hat{=} \;}}
\newcommand{\FOL}{\ensuremath{\textup{\tiny{}FOL}}}
\newcommand{\false}{\textup{ff}}
\newcommand{\true}{\textup{tt}}
\newcommand{\M}{\ensuremath{\mathcal{M}}}
\newcommand{\I}{\ensuremath{\mathcal{I}}}
\newcommand{\dom}{\ensuremath{\textup{ dom }}}

\usepackage{xcolor}
\newcommand{\raph}[1]{\textcolor{red}{#1}}

\definecolor{myblue}{rgb}{0,0.5,0.8}
\newcommand{\mpar}[1]{\marginpar{\color{red}\footnotesize\raggedright#1}}
\newcommand{\bpar}[1]{\marginpar{\color{myblue}\footnotesize\raggedright#1}}
\addtolength{\marginparwidth}{1cm}
\long\def\stephan#1{{\color{myblue} #1}}
\newcommand{\TRUE}{\textsc{true}}
\newcommand{\FALSE}{\textsc{false}}

\newtheorem{prop}{Propriété}
\newtheorem{lem}{Lemme}

\begin{document}

\maketitle

On étudie dans ce document le remplacement d'un opérateur par sa définition dans une formule FOML.
On cherche à montrer en particulier que pour un opérateur $d(\vec{x_i}) \triangleq e_d$ et une formule FOML $\phi$, si $\phi$ est valide une fois abstraite en FOL, alors la formule $\widetilde{\phi}$ obtenue en remplaçant chaque occurence de $d(\vec{e_i})$ par $e_d(\vec{e_i}/\vec{x_i})$ est également valide une fois abstraite en FOL.

On cherche alors à établir une correspondance sémantique entre les abstractions FOL de $\phi$ et $\widetilde{\phi}$.
\[
  \phi \to \phi_\FOL
\]
\[
  \phi \to \widetilde{\phi} \to \widetilde{\phi}_\FOL
\]

\raph{%
  (J'effacerai ce schéma ou j'en ferai un mieux, mais pour l'instant il permet de comprendre quelles formules qu'on manipule.)}

\paragraph{Définition} Remplacement d'un opérateur défini\\
\bpar{On suppose que $d$ n'apparaît pas dans $e_d$?}
Soit $d(\vec{x_i}) \triangleq e_d$ un opérateur défini, et $\phi$ une formule FOML pouvant contenir des occurences de $d$.
On définit récursivement $\widetilde{\phi}$ la formule obtenue en remplaçant $d$ par sa définition dans $\phi$ par :
\begin{align*}
  \widetilde{x} &\defeq x \\
  \widetilde{v} &\defeq v \\
  \widetilde{op(\vec{e_i})} &\defeq op(\vec{\widetilde{e_i}}) 
  \qquad\text{\stephan{si $op$ différent de $d$}}\\
  \widetilde{e_1 = e_2} &\defeq \widetilde{e_1} = \widetilde{e_2} \\
  \widetilde{\textup{FALSE}} &\defeq \textup{FALSE} \\
  \widetilde{e_1 \Rightarrow e_2} &\defeq \widetilde{e_1} \Rightarrow \widetilde{e_2} \\
  \widetilde{\forall x : e } &\defeq \forall x : \widetilde{e} \\
  \widetilde{\nabla e} &\defeq \nabla \widetilde{e} \\
  \widetilde{d(\vec{e_i})} &\defeq e_d(\vec{\widetilde{e_i}}/\vec{x_i})
\end{align*}

\raph{%
  Cette définition correspond bien à ce qu'on veut faire.\\
  Remarque 1 : Ici on n'a pas parlé des opérateurs définis autres que d.\\
  Remarque 2 : On ne remplace pas «qu'une fois» $d$, mais récursivement.
  Par ex $d(d(e))$ donne $e_d([e_d(e/x)]/x)$ et pas $e_d([d(e)]/x)$.}

\stephan{%
  $e_d(\vec{\widetilde{e_i}}/\vec{x_i})$ plutôt que $\widetilde{e_d}(\vec{\widetilde{e_i}}/\vec{x_i})$ suffit car $d$ ne peut apparaître dans $e_d$ et sinon ça ne terminerait pas. S'il y a plusieurs opérateurs définis (toujours sans récursion) il faudra les éliminer l'un après l'autre.}

\begin{prop}
  \label{prop-sem}
Soit $d(\vec{x_i}) \triangleq e_d$ un opérateur défini, et $\phi$ une formule FOML pouvant contenir des occurences de $d$.
Si $\vDash_\FOL \phi_\FOL$ alors $\vDash_\FOL \widetilde{\phi}_\FOL$.
\end{prop}

\paragraph{Preuve de la propriété \ref{prop-sem}}
Par complétude et correction de la déduction naturelle, il suffit de montrer la propriété \ref{prop-syn}.

\begin{prop}
  \label{prop-syn}
  Pour toute formule FOML $\phi$ et tout ensemble de formules FOML $\Gamma$,
  si $\Gamma_\FOL \vdash \phi_\FOL$, alors $\widetilde{\Gamma}_\FOL \vdash \widetilde{\phi}_\FOL$.
\end{prop}

\paragraph{Preuve de la propriété \ref{prop-syn}}
On fait une preuve par récurrence sur la dérivation de $\Gamma_\FOL \vdash \phi_\FOL$ en considérant la dernière règle étudiée.

La logique FOL étudiée ne contient pas les symboles $\land, \lor, \exists, \neg$.
On étudiera seulement les règles suivantes : ax, RAA, $\Rightarrow_I$, $\Rightarrow_E$, $\forall_I$, $\forall_E$, $\bot_E$.

On montre les cas RAA, $\Rightarrow_I$, $\Rightarrow_E$, $\bot_E$ en appliquant les définitions et l'hypothèse de récurrence.
Le cas ax est trivial.
On considère maintenant les cas $\forall_I$ et $\forall_E$.

\begin{itemize}
\item
  Cas $\forall_I$ :\\
  On suppose qu'il existe une preuve de
  \begin{prooftree}
    \hypo{ \Gamma_\FOL \vdash \psi }
    \infer1[$\forall_I$]
    { \Gamma_\FOL \vdash \phi_\FOL = \forall x : \psi }
  \end{prooftree}, où $x \notin fv(\Gamma_\FOL)$.
  
  Par analyse de cas sur la définition de $\phi_\FOL$ on trouve que $\phi = \forall x : \phi_1$ avec ${\phi_1}_\FOL = \psi$ :
  \begin{prooftree}
    \hypo{ \Gamma_\FOL \vdash {\phi_1}_\FOL }
    \infer1[$\forall_I$]
    { \Gamma_\FOL \vdash \forall x : {\phi_1}_\FOL }
  \end{prooftree}

  On peut alors appliquer l'hypothèse de récurrence, il existe donc une dérivation de $\widetilde{\Gamma}_\FOL \vdash \widetilde{\phi_1}_\FOL$.
  
  On peut montrer que si $x \notin fv(e_\FOL)$, alors $x \notin fv(\widetilde{e}_\FOL)$ par récurrence sur $e$. On en déduit que $x \notin fv(\widetilde{\Gamma}_\FOL)$.

  On applique donc la règle $\forall_I$ pour obtenir la dérivation suivante :
  \begin{prooftree}
    \hypo{ \widetilde{\Gamma}_\FOL \vdash \widetilde{\phi_1}_\FOL }
    \infer1[$\forall_I$]
    { \widetilde{\Gamma}_\FOL \vdash \forall x : \widetilde{\phi_1}_\FOL } 
  \end{prooftree}
  où $\forall x : \widetilde{\phi_1}_\FOL = \widetilde{(\forall x : \phi_1)}_\FOL = \widetilde{\phi}_\FOL$.


  \vspace{1cm}
\item
  Cas $\forall_E$ : \raph{C'est ce cas qui est le plus dur} \\
  On suppose qu'il existe une preuve de
  \begin{prooftree}
    \hypo{ \Gamma_\FOL \vdash \forall x : \psi }
    \infer1[$\forall_E$]
    { \Gamma_\FOL \vdash \phi_\FOL = \psi\{x \mapsto t\} }
  \end{prooftree}.

  \raph{%
    On aimerait pouvoir appliquer l'hypothèse de récurrence sur $\Gamma_\FOL \vdash \forall x : \psi$. \\
    Pour ça il suffirait de montrer qu'il existe $\theta$ tel que $\theta_\FOL = \psi$.
    On veut montrer le lemme \ref{lemme-theta}. \\
    Le problème est que ce n'est pas toujours le cas.}
\end{itemize}

\begin{lem}
  \label{lemme-theta}
  Soit $\phi$ une formule FOML, $\psi, t$ des formules FOL telles que $\phi_\FOL = \psi\{x \mapsto t\}$.
  Alors il existe $\theta, \zeta$ des formules FOML telles que :
  \begin{enumerate}
  \item
    $\theta_\FOL = \psi$
  \item
    $\zeta_\FOL = t$ ou $x \notin fv(\psi)$
  \item
    $\zeta$ est rigide, ou $x$ n'apparait qu'à des positions Leibniz dans $\theta$
  \end{enumerate}
  \raph{Problème : ce lemme est faux, voir à la fin du document}\\
  \raph{Solution : ce lemme est faux mais pour un cas qui ne nous intéresse pas dans la preuve de la propriété \ref{prop-syn}.}
\end{lem}

\paragraph{Preuve du lemme \ref{lemme-theta}}

\raph{On n'arrivera pas à montrer tous les cas.}\\
On fait une preuve par récurrence sur $\psi$ et on montre la propriété avec $\phi_\FOL^z$ pour tout $z$.

On arrive effectivement à montrer les cas où $\psi = v, y (\neq x), x, \textup{FALSE}$.
La preuve fonctionne aussi pour les cas où $\psi = op(\vec{\psi_i}), \psi_1 \Rightarrow \psi_2, \psi_1 = \psi_2, \forall x : \psi_1$
La preuve échoue pour le cas \(\psi = \framebox{$d_{\vec{\epsilon}}$}(\vec{\psi_i}) \).

\paragraph{Contre-exemple au lemme \ref{lemme-theta}}

On considère un opérateur défini $d$ prenant un argument dont la position n'est pas Leibniz. \\
On pose :
\[ \phi = d(v) \]
Alors \(\phi_\FOL = \framebox{$d_v$}(v) = (\framebox{$d_v$}(x))\{x \mapsto v\} \).
On pose :
\[ \psi = \framebox{$d_v$}(x) \]
\[ t = v \]
On peut montrer qu'il n'existe pas de formule FOML $\theta$ telle que $\theta_\FOL = \psi$.

\vspace{1cm}

\raph{%
  Le contre exemple ci-dessus semble être un cas \og{}aberrant\fg dans la preuve de la propriété \ref{prop-syn},
  en effet le séquent \(\Gamma_\FOL \vdash \forall x : \framebox{$d_v$}(x) \) ne semble pas valide (sauf si $\Gamma_\FOL$ est incohérente).\\
  On montre effectivement que si le séquent précédent est dérivable, alors $\Gamma_\FOL$ est incohérente.}

\vspace{1cm}

\begin{lem}
  \label{lemme-T}
  Soient $\Gamma_\FOL$ un ensemble d'abstractions de formules FOML,
  $d$ un opérateur défini unaire et dont la position n'est pas Leibniz,
  $e$ une formule FOML flexible.\\
  On suppose que \(\Gamma_\FOL \vDash \forall x : \framebox{$d_e$}(x) \).\\
  On montre que $\Gamma_\FOL$ est incohérente.
\end{lem}

Pour montrer le lemme \ref{lemme-T} on aura besoin du lemme suivant :

\begin{lem}
  \label{lemme-S}
  Soient \M un modèle FOL,
  $d$ un opérateur défini unaire et dont la position n'est pas Leibniz,
  $e$ une formule FOML flexible,
  Pour tous $\phi$ formule FOML, $y$ ensemble de variables rigides,
  la valuation $\llbracket \phi_\FOL^y \rrbracket_\M$ ne dépend pas de \(\I(\framebox{$d_e$})(a)\) pour
  $a \in \dom \M$ tel que $a \neq \llbracket \phi_\FOL^{y'} \rrbracket_\M$ pour tout $y' \subseteq y$.
\end{lem}

\paragraph{Preuve du lemme \ref{lemme-S}}

On fait une preuve par récurrence sur $\phi$.

Les cas $\phi = x, v, FALSE$ sont immédiats.
De même les cas $\phi = \phi_1 \Rightarrow \phi_2, \phi_1 = \phi_2, op(\vec{\phi_i})$ sont immédiats en appliquant l'hypothèse de récurrence.
On parvient également à montrer les autres cas. \raph{Je n'ai pas fini de recopier, mais on peut finir la preuve}

\paragraph{Preuve du lemme \ref{lemme-T}}

Soit \M un modèle FOL de $\Gamma_\FOL$, où on peut interpréter $e_\FOL$.\\
On pose $\M'$ qui coincide avec \M, sauf pour l'interprétation de \framebox{$d_e$} :
\[ \I'(\framebox{$d_e$})(a) =
  \left \{
    \begin{array}{l l}
      \I(\framebox{$d_e$})(a) & \textup{ si } a = \llbracket e_\FOL \rrbracket_\M \\
      \false & \textup{ sinon }
    \end{array}
  \right . \]
Par le lemme \ref{lemme-S} appliqué aux formules de $\Gamma_\FOL$, $\M'$ est également un modèle de $\Gamma_\FOL$. \\
De plus \(\M' \nvDash \forall x : \framebox{$d_e$}(x)\) car $\dom \M'$ comporte au moins deux éléments ($\true$ et $\false$), pour l'un desquels \(\I'(\framebox{$d_e$})\) affecte $\false$.

Ceci est absurde car on a supposé que \(\Gamma_\FOL \vDash \forall x : \framebox{$d_e$}(x)\).
Alors $\Gamma_\FOL$ n'a aucun modèle et est incohérente par complétude.

\vspace{1cm}

\raph{%
  On a donc montré que dans la preuve de la propriété \ref{prop-syn},
  lorsque le cas proposé dans le contre-exemple apparaît,
  appliquer le lemme est en fait inutile puisque la théorie $\Gamma_\FOL$ est incohérente.
  On peut alors montrer que $\widetilde{\Gamma}_\FOL$ est également incohérente.}

\begin{lem}
  \label{lemme-theta-corr}
  \raph{correction du lemme \ref{lemme-theta}}\\
  Soit $\phi$ une formule FOML, $\psi, t$ des formules FOL telles que $\phi_\FOL = \psi\{x \mapsto t\}$.\\
  On exclut le cas où $\psi$ contient une sous formule \(\framebox{$d_{\vec{\epsilon}}$}(\vec{\psi'_i})\)
  où il existe $i$ tel que $\psi'_i$ est engendré par une formule rigide et $\epsilon_i \neq *$. \\
  Alors il existe $\theta, \zeta$ des formules FOML telles que :
  \begin{enumerate}
  \item
    $\theta_\FOL = \psi$
  \item
    $\zeta_\FOL = t$ ou $x \notin fv(\psi)$
  \item
    $\zeta$ est rigide, ou $x$ n'apparait qu'à des positions Leibniz dans $\theta$
  \end{enumerate}
\end{lem}

\vspace{1cm}

\raph{%
  On remarque que le contre-exemple ci-dessus survient à cause de la manière dont sont abstraits les
  opérateurs définis. \\
  Définir
  \[d(v) \defeq \framebox{$d_v$}(v)\]
  est redondant car $v$ apparait deux fois.
  Dans le contre-exemple, on peut alors construire une formule FOL ne pouvant pas être engendrée par l'abstraction d'une formule FOML. \\
  Une solution serait de poser une définition différente pour l'abstraction d'un opérateur défini, où les arguments flexibles à une position non Leibniz ne seraient plus \og{}accessibles\fg en FOL, par exemple
  \[d(v) \defeq \framebox{$d_v$}\]
}

\end{document}
