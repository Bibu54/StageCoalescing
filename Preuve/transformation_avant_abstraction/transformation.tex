\documentclass[12pt]{article}

\usepackage[utf8]{inputenc}
\usepackage[T1]{fontenc}
\usepackage[french]{babel}
\usepackage{amsmath, amssymb}
\usepackage{stmaryrd}
\usepackage{comment}
\usepackage{ulem}


\title{Transformation de formule avant abstraction}
\author{Raphaël Le Bihan}

\newcommand{\defeq}{\ensuremath{\; \hat{=} \;}}
\newcommand{\FOL}{\ensuremath{\textup{\tiny{}FOL}}}
\newcommand{\false}{\textup{ff}}
\newcommand{\true}{\textup{tt}}
\newcommand{\M}{\ensuremath{\mathcal{M}}}
\newcommand{\I}{\ensuremath{\mathcal{I}}}
\newcommand{\dom}{\ensuremath{\textup{ dom }}}
\newcommand{\lra}{\ensuremath{\longrightarrow}}
\newcommand{\qbq}{\ensuremath{\quad | \quad}}

\usepackage{xcolor}
\newcommand{\raph}[1]{\textcolor{red}{#1}}

\definecolor{myblue}{rgb}{0,0.5,0.8}
\newcommand{\mpar}[1]{\marginpar{\color{red}\footnotesize\raggedright#1}}
\newcommand{\bpar}[1]{\marginpar{\color{myblue}\footnotesize\raggedright#1}}
\addtolength{\marginparwidth}{1cm}
\long\def\stephan#1{{\color{myblue} #1}}
\newcommand{\TRUE}{\textsc{true}}
\newcommand{\FALSE}{\textsc{false}}

\newtheorem{prop}{Propriété}
\newtheorem{defin}{Définition}

\begin{document}

\maketitle

Ce document développe l'idée de transformer les formules en TLA$^+$ avant de les abstraire, présentée dans le fichier \texttt{TestAbstracion.tla}.

\paragraph{Observations}

Certaines formules valides en TLA$^+$ ne sont pas démontrables en utilisant seulement l'abstraction vers la FOL.
On rencontre plusieurs cas dans les preuves en TLA$^+$ qu'il serait utile de pouvoir démontrer.

Un premier cas est la formule de Barcan $(\forall x : \Box P(x)) \Rightarrow (\Box \forall x : P(x))$ et sa contraposée.

Un deuxième cas utile pour des preuves en TLA$^+$ est le suivant :
on souhaite prouver l'objectif de preuve $\Box B$ à partir de l'hypothèse $\Box A$, où $A \Rightarrow B$ est valide sans autre hypothèse.
Ceci est correct car pour une formule $A \Rightarrow B$ valide, $\Box A \Rightarrow \Box B$ est également valide.
L'abstraction (vers la FOL ou la ML) n'est cependant pas toujours suffisante pour prouver une telle obligation.

Par exemple lorsque
\[ \Box A = \Box \forall x : \neg (A_1 \lor A_2) \]
\[ \Box B = \Box \forall x : \neg A_1 \land \neg A_2\]
lors de l'abstraction vers la FOL, les deux opérateurs $\Box$ sont abstraits vers des opérateurs FOL différents ; de même lors de l'abstraction vers la ML, les deux opérateurs $\forall x$ sont abstraits vers des propositions ML différentes.

On ne peut alors montrer l'obligation de preuve ni avec OBVIOUS ni avec BY PTL.

\begin{comment}
Remarque : Ici on raisonnera sur des formules TLA$^+$ et pas sur des formules FOML.
Dans la FOML l'interprétation du symbole $\nabla$ n'est pas connue,
les transformations que l'on peut effectuer sur une formule FOML tout
en gardant des formules équivalentes sont donc moindres par rapport aux transformations sur les formules TLA$^+$.
\end{comment}

\paragraph{Transformation de formule avant abstraction}

Une solution envisagée est de transformer des formules FOML en autres formules FOML équivalentes.
Cette transformation devrait préserver l'équivalence entre la formule initiale et la formule finale et permettre d'abstraire différentes formules FOML vers des opérateurs FOL ou propositions ML identiques.

Dans l'exemple précédent, les formules $\neg (A_1 \lor A_2)$ et $\neg A_1 \land \neg A_2$ équivalentes devraient être transformées vers une même formule FOML $\phi$.
L'abstraction (vers la FOL ou la ML) devrait alors abstraire $\Box A$ et $\Box B$ vers la même formule FOL et permettre de prouver l'obligation de preuve.


\paragraph{Système de réécriture}

On définit la transformation effectuée sous forme d'un système de réécriture.

\begin{defin}
  On définit les différentes règles de réécriture pour une formule TLA$^+$.
  \begin{align*}
    \Diamond \phi & \longrightarrow \neg \Box \neg \phi \\
    \exists x : \phi & \longrightarrow \neg \forall x : \neg \phi \\
    \neg \neg \phi & \longrightarrow \phi \\
    \phi_1 \Leftrightarrow \phi_2 & \longrightarrow (\phi_1 \Rightarrow \phi_2) \land (\phi_2 \Rightarrow \phi_1) \\
    \phi_1 \Rightarrow \phi_2 & \longrightarrow \neg \phi_1 \lor \phi_2 \\
    \phi_1 \lor \phi_2 & \longrightarrow \neg(\neg \phi_1 \land \neg \phi_2) \\
                  & \textup{\raph{J'ai changé cette règle pour que les \(\lor\)}}\\
                  & \textup{\raph{soient éliminés plutôt que les \(\land\)}}\\
    \TRUE & \longrightarrow \neg \FALSE \\
    \Box \Box \phi & \longrightarrow \Box \phi \\
    \Box (\phi_1 \land \phi_2) & \lra (\Box \phi_1) \land (\Box \phi_2) \textup{\raph{Ajout 1}}\\
    (\phi_1 \leadsto \phi_2) & \lra \Box (\phi_1 \Rightarrow \Diamond \phi_2) \textup{\raph{Ajout 2}}
  \end{align*}
  Chaque règle peut être appliquée au sein d'une formule.
  On notera alors \( \psi_i \longrightarrow \psi_f \) où \(\psi_f\) est obtenue en appliquant une étape de réécriture sur une sous-formule de \(\psi_i\).
\end{defin}

La définition suivante sera utile pour formaliser les preuves des différentes propriétés.

\begin{defin} \emph{Contexte} \\
  Un contexte \( C[\_] \) est une modèle de formule contenant un emplacement libre dans lequel on peut insérer une sous-formule \( \phi \).
  Ceci forme alors la formule \( \psi = C[\phi] \)
\end{defin}

On travaillera avec la formalisation suivante pour les formules TLA$^+$.

\begin{defin}
  Une formule TLA$^+$ est engendrée par la grammaire suivante :
  \begin{align*}\phi ::= & \quad
    \TRUE \quad | \quad
    \FALSE \quad | \quad
    \Box \phi \quad | \quad
    \Diamond \phi \quad | \quad
    \exists x : \phi \quad | \quad
    \forall x : \phi \quad | \quad
                           \phi \land \phi \\
                         & \quad | \quad
    \phi \lor \phi \quad | \quad
    \neg \phi \quad | \quad
    \phi \Rightarrow \phi \quad | \quad
    \phi \Leftrightarrow \phi \quad | \quad
    \phi \leadsto \phi \quad | \quad
      P
  \end{align*}
  pour tout symbole de prédicat $P$.
\end{defin}

\paragraph{Remarques}

On pourrait essayer ensuite d'étendre ce système de réécriture, par exemple en développant les formules du type $A \land (B \lor C) \longrightarrow (A \land C) \lor (A \land B)$ en s'inspirant du calcul d'une forme clausale pour une formule FOL.
Cela pourrait permettre d'identifier plus de formules vers une même formule TLA$^+$ mais risque d'impliquer un temps exponentiel lors du calcul.


\paragraph{Propriétés}

On montre différentes propriétés vérifiées par ce système de réécriture

\begin{prop} \emph{Correction du système de réécriture} \\
  Pour $\psi_i, \psi_f$ des formules TLA$^+$ telles que $\psi_i \longrightarrow^* \psi_f$,
  la formule FOML $\psi_i \Leftrightarrow \psi_f$ est valide.
\end{prop}

On peut démontrer cette propriété par récurrence sur la longueur de la dérivation.
Le cas où $\psi_i = \psi_f$ est immédiat. Il suffit de vérifier que la propriété est vraie dans le cas où $\psi_i \longrightarrow \psi_f$.
On remarque que dans chaque étape de réécriture les formules à gauche et à droite sont équivalentes.
L'équivalence est préservée en insérant ces formules dans un contexte,
ce qui conclut la démonstration.

\raph{ok}

\begin{prop} \emph{Terminaison du système de réécriture} \\
  Pour $\psi$ une formule TLA$^+$ il n'existe pas de dérivation infinie partant de $\psi$.
\end{prop}

On peut le montrer en utilisant un ordre bien fondé sur les formules TLA$^+$ tel que pour chaque étape de réécriture la formule finale soit strictement inférieure à la formule initiale.

Pour $\phi$ une formule FOML, on définit sa hauteur \( h(\phi) \) récursivement par :
\begin{align*}
  h(\neg \phi) & = h(\phi) + 1 \\
  h(\Box \phi) & = 2h(\phi) \\
  h(\Diamond \phi) & = 2h(\phi) + 4 \\
  h(\forall x : \phi) & = h(\phi) \\
  h(\exists x : \phi) & = h(\phi) + 3 \\
  h(\FALSE) & = 0 \\
  h(\TRUE) & = 2 \\
  h(P) & = 0 \textup{ pour tout prédicat \(P\) } \\
  h(\phi_1 \land \phi_2) & = h(\phi_1) + h(\phi_2) + 1\\
  h(\phi_1 \lor \phi_2) & = h(\phi_1) + h(\phi_2) + 5\\
  h(\phi_1 \Rightarrow \phi_2) & = h(\phi_1) + h(\phi_2) + 7\\
  h(\phi_1 \Leftrightarrow \phi_2) & = 2h(\phi_1) + 2h(\phi_2) + 16\\
  h(\phi_1 \leadsto \phi_2) & = 2h(\phi_1) + 4h(\phi_2) + 23
\end{align*}
On peut alors montrer que le système de réécriture vérifie les trois propriétés suivantes :
\begin{enumerate}
\item La hauteur de toute formule est entière positive.
\item Pour toute règle de réécriture, la hauteur de la formule de droite est strictement plus faible que la hauteur de la formule de gauche.
\item La hauteur d'une formule croit strictement avec la hauteur de ses sous formules.
  Plus formellement, pour tout contexte \( C[\_] \) et pour toutes formules \( \phi_1, \phi_2 \), si \( h(\phi_1) < h(\phi_2) \) alors \( h(C[\phi_1]) < h(C[\phi_2]) \).
\end{enumerate}

La troisième propriété est utile car on autorise la réécriture d'une formule en appliquant une étape parmi ses sous-formules.
Avec ces deux propriétés on peut montrer que pour toutes formules \( \psi_i \longrightarrow^+ \psi_f \), \( h(\psi_i) > h(\psi_f) \).
Ceci montre la terminaison car l'ordre strict sur les entiers naturels est un ordre bien fondé.

\begin{prop} \emph{Confluence de la réécriture} \\
  Pour \( \psi_i, \psi_a, \psi_b \) des formules telles que \( \psi_i \longrightarrow^* \psi_a \) et \( \psi_i \longrightarrow^* \psi_b \) alors il existe \( \psi_f \) telle que \( \psi_a \longrightarrow^* \psi_f \) et \( \psi_b \longrightarrow^* \psi_f \).
\end{prop}

Comme la réécriture termine, il suffit de montrer la confluence faible :

\begin{prop} \emph{Confluence faible de la réécriture} \\
  \label{confl_faible}
  Pour \( \psi_i, \psi_a, \psi_b \) des formules telles que \( \psi_i \longrightarrow \psi_a \) et \( \psi_i \longrightarrow \psi_b \) alors il existe \( \psi_f \) telle que \( \psi_a \longrightarrow^* \psi_f \) et \( \psi_b \longrightarrow^* \psi_f \).
\end{prop}


\paragraph{Preuve de la propriété \ref{confl_faible}}
On note tout d'abord \(\phi_a^i \longrightarrow \phi_a\) et \(\phi_b^i \longrightarrow \phi_b\) les règles appliquées dans les étapes de transformation \(\psi_i \lra \psi_a\) et \(\psi_i \lra \psi_b\).
\(\phi_a^i\) et \(\phi_b^i\) sont donc des sous-formules de \(\psi_i\), \(\phi_a\) de \(\psi_a\) et \(\phi_b\) de \(\psi_b\).
\begin{comment}
On pose tout d'abord les notations :
\[ \psi_i = C_a[\phi_a^i] \longrightarrow C_a[\phi_a] = \psi_a \]
\[ \psi_i = C_b[\phi_b^i] \longrightarrow C_b[\phi_b] = \psi_b \]
où les règles appliquées sont \( \phi_a^i \longrightarrow \phi_a \) et \( \phi_b^i \longrightarrow \phi_b \).
\end{comment}

% On fait une disjonction de cas sur la règle appliquée dans \( \phi_a^i \longrightarrow \phi_a \).
% Par symétrie on suppose que la règle appliquée sur \( \phi_b^i \longrightarrow \phi_b \) n'est pas l'élimination d'un symbole \(\Leftrightarrow\).

\smallskip
On considère d'abord le cas où \( \phi_a^i \) n'est pas une sous-formule de \( \phi_b^i \) dans  \(\psi_i\) et inversement.
Alors \( \psi_i\) est de la forme \(\psi_i = C[\phi_a^i, \phi_b^i]\) où \(C[\_,\_]\) est un contexte à deux emplacements, avec \(\psi_a = C[\phi_a, \phi_b^i]\) et \(\psi_b = C[\phi_a^i, \phi_b]\).
On pose alors \(\psi_f = C[\phi_a, \phi_b]\), qui peut être obtenue à partir de \(\psi_a\) en appliquant l'étape \(\phi_b^i \longrightarrow \phi_b \) ; de même pour \(\psi_b\).


\smallskip
On considère ensuite l'autre cas où parmi \(\phi_a^i\) et \(\phi_b^i\) l'une des formule est une sous-formule de l'autre.
Par symétrie on suppose que \(\phi_b^i\) est une sous-formule de \(\phi_a^i\).

On remarque tout d'abord par définition des règles de réécriture que si \(\phi_a^i = \phi_b^i\) alors la règle appliquée est la même dans \(\phi_a^i \lra \phi_a\) et \(\phi_b^i \lra \phi_b\).
La propriété est vérifiée dans ce cas, on considèrera donc dans la suite de la preuve que \(\phi_a^i \neq \phi_b^i\).

On remarque ensuite que \(\phi_a^i\) est de la forme \(\phi_a^i = C[\phi_b^i]\), pour vérifier la propriété il suffit donc de trouver une formule \(\phi_f\) telle que \(\phi_a^i \lra \phi_a \lra \phi_f\) et \(C[\phi_b^i] \lra C[\phi_b] \lra \phi_f\). C'est ce qu'on cherchera à montrer dans la suite de la preuve.

\[ \begin{array}{c c c}
     \phi_a^i = C[\phi_b^i] & \lra & C[\phi_b^i] \\
     \downarrow & & \downarrow \\
     \phi_a & \lra & \phi_f
   \end{array} \]

On fait une disjonction de cas sur \(\phi_a^i\).

\begin{itemize}
\item Cas \(\phi_a^i = \Diamond \phi_a^j\) : \\
  La règle qui est appliquée pour \(\phi_a^i \lra \phi_a\) est alors \(\Diamond \phi_a^j \lra \neg \Box \neg \phi_a^j\).
  De plus \(\phi_b^i\) est une sous-formule de \(\phi_a^j\), on notera \(\phi_a^j = C_b[\phi_b^i]\).
  On pose \(\phi_f = \neg \Box \neg C_b[\phi_b]\).
  Alors \(\phi_a = \neg \Box \neg C_b[\phi_b^i] \lra \neg \Box \neg C_b[\phi_b] = \phi_f\), et \(\Diamond C_b[\phi_b] \lra \neg \Box \neg C_b[\phi_b] = \phi_f\).

\item Le {cas \(\phi_a^i = \exists x : \phi_a^j\)} est similaire.
\item Cas \(\phi_a^i = \neg \phi_a^j\) : \\
  Par élimination sur les règles de réécriture la règle appliquée pour \(\phi_a^i \lra \phi_a\) est \(\neg \neg \phi \lra \phi\) donc \(\phi_a^i\) est de la forme \(\phi_a^i = \neg \neg \phi_a^k\).

  Si \(\phi_b^i\) est une sous-formule de \(\phi_a^k\) on peut utiliser la même preuve que pour les cas précédents.
  
  Sinon \(\phi_a^i = \neg \phi_b^i = \neg \neg \phi_a^k\).
  De même on peut déduire que la règle appliquée pour \(\phi_b^i \lra \phi_b\) est \(\neg \neg \phi \lra \phi\) et donc \(\phi_b^i = \neg \neg \phi_a^l\).

  Alors
  \[\phi_a^i = \neg \neg \phi_a^k \lra \phi_a^k = \phi_a\]
  et
  \[\phi_a^i = \neg \phi_b^i = \neg (\neg \neg \phi_a^l) \lra \neg \phi_a^l = \neg \phi_b\]
  Finalement \(\phi_a = \neg \phi_b\) ce qui conclut la preuve pour ce cas.
\item Cas \(\phi_a^i = \phi_a^{j1} \Leftrightarrow \phi_a^{j2}\) : \\
  On suppose que \(\phi_b\) est une sous formule de \(\phi_a^{j1}\) et on note \(\phi_a^{j1} = C_b[\phi_b^i]\).
  Par élimination sur les différentes règles la règle qui est appliquée pour \(\phi_a^i \lra \phi_a\) est
  \[\phi_a^i = C_b[\phi_b^i] \Leftrightarrow \phi_a^{j2} \lra (C_b[\phi_b^i] \Rightarrow \phi_a^{j2}) \land (\phi_a^{j2} \Rightarrow C_b[\phi_b^i]) = \phi_a\]
  On pose \(\phi_f = (C_b[\phi_b] \Rightarrow \phi_a^{j2}) \land (\phi_a^{j2} \Rightarrow C_b[\phi_b]) = \phi_a\).
  Alors
  \[\begin{array}{c c c}
      C_b[\phi_b^i] \Leftrightarrow \phi_a^{j2} & \lra & C_b[\phi_b] \Leftrightarrow \phi_a^{j2} \\
      \downarrow & & \downarrow \\
      (C_b[\phi_b^i] \Rightarrow \phi_a^{j2}) & & (C_b[\phi_b] \Rightarrow \phi_a^{j2}) \\
      \land (\phi_a^{j2} \Rightarrow C_b[\phi_b^i]) & \lra^2 & \land (\phi_a^{j2} \Rightarrow C_b[\phi_b])
    \end{array} \]
  Le cas où \(\phi_b^i\) est une sous-formule de \(\phi_a^{j2}\) est similaire.
\item Les cas \(\phi_a^i =
  \phi_a^{j1} \leadsto \phi_a^{j2} \qbq
  \phi_a^{j1} \Rightarrow \phi_a^{j2} \qbq
  \phi_a^{j1} \lor \phi_a^{j2}\)
  sont similaires.
\item Cas \(\phi_a^i = \Box \phi_a^j\) : \\
  \raph{à finir de rédiger}
\item
  Tous les autres cas sont absurdes, soit parce que \(\phi_a^i\) n'a pas de sous-formule stricte soit parce qu'aucune règle de réécriture ne peut s'appliquer.
\end{itemize}


\end{document}