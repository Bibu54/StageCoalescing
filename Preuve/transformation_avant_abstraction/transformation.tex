\documentclass[12pt]{article}

\usepackage[utf8]{inputenc}
\usepackage[T1]{fontenc}
\usepackage[french]{babel}
\usepackage{amsmath, amssymb}
\usepackage{stmaryrd}
\usepackage{comment}
\usepackage{ulem}


\title{Transformation de formule avant abstraction}
\author{Raphaël Le Bihan}

\newcommand{\defeq}{\ensuremath{\; \hat{=} \;}}
\newcommand{\FOL}{\ensuremath{\textup{\tiny{}FOL}}}
\newcommand{\false}{\textup{ff}}
\newcommand{\true}{\textup{tt}}
\newcommand{\M}{\ensuremath{\mathcal{M}}}
\newcommand{\I}{\ensuremath{\mathcal{I}}}
\newcommand{\dom}{\ensuremath{\textup{ dom }}}

\usepackage{xcolor}
\newcommand{\raph}[1]{\textcolor{red}{#1}}

\definecolor{myblue}{rgb}{0,0.5,0.8}
\newcommand{\mpar}[1]{\marginpar{\color{red}\footnotesize\raggedright#1}}
\newcommand{\bpar}[1]{\marginpar{\color{myblue}\footnotesize\raggedright#1}}
\addtolength{\marginparwidth}{1cm}
\long\def\stephan#1{{\color{myblue} #1}}
\newcommand{\TRUE}{\textsc{true}}
\newcommand{\FALSE}{\textsc{false}}

\newtheorem{prop}{Propriété}
\newtheorem{defin}{Définition}

\begin{document}

\maketitle

Ce document développe l'idée de transformer les formules en TLA$^+$ avant de les abstraire, présentée dans le fichier \texttt{TestAbstracion.tla}.

\paragraph{Observations}

Certaines formules valides en TLA$^+$ ne sont pas démontrables en utilisant seulement l'abstraction vers la FOL.
On rencontre plusieurs cas dans les preuves en TLA$^+$ qu'il serait utile de pouvoir démontrer.

Un premier cas est la formule de Barcan $(\forall x : \Box P(x)) \Rightarrow (\Box \forall x : P(x))$ et sa contraposée.

Un deuxième cas utile pour des preuves en TLA$^+$ est le suivant :
on souhaite prouver l'objectif de preuve $\Box B$ à partir de l'hypothèse $\Box A$, où $A \Rightarrow B$ est valide sans autre hypothèse.
Ceci est correct car pour une formule $A \Rightarrow B$ valide, $\Box A \Rightarrow \Box B$ est également valide.
L'abstraction (vers la FOL ou la ML) n'est cependant pas toujours suffisante pour prouver une telle obligation.

Par exemple lorsque
\[ \Box A = \Box \forall x : \neg (A_1 \lor A_2) \]
\[ \Box B = \Box \forall x : \neg A_1 \land \neg A_2\]
lors de l'abstraction vers la FOL, les deux opérateurs $\Box$ sont abstraits vers des opérateurs FOL différents ; de même lors de l'abstraction vers la ML, les deux opérateurs $\forall x$ sont abstraits vers des propositions ML différentes.

On ne peut alors montrer l'obligation de preuve ni avec OBVIOUS ni avec BY PTL.

\begin{comment}
Remarque : Ici on raisonnera sur des formules TLA$^+$ et pas sur des formules FOML.
Dans la FOML l'interprétation du symbole $\nabla$ n'est pas connue,
les transformations que l'on peut effectuer sur une formule FOML tout
en gardant des formules équivalentes sont donc moindres par rapport aux transformations sur les formules TLA$^+$.
\end{comment}

\paragraph{Transformation de formule avant abstraction}

Une première solution envisagée est de transformer des formules FOML en autres formules FOML équivalentes.
Cette transformation devrait préserver l'équivalence entre la formule initiale et la formule finale et permettre d'abstraire différentes formules FOML vers des opérateurs FOL ou propositions ML identiques.

Dans l'exemple précédent, les formules $\neg (A_1 \lor A_2)$ et $\neg A_1 \land \neg A_2$ équivalentes devraient être transformées vers une même formule FOML $\phi$.
L'abstraction (vers la FOL ou la ML) devrait alors abstraire $\Box A$ et $\Box B$ vers la même formule FOL et permettre de prouver l'obligation de preuve.


\paragraph{Système de réécriture}

On définit la transformation effectuée sous forme d'un système de réécriture.

\begin{defin}
  On définit les différentes règles de réécriture pour une formule TLA$^+$.
  Chaque règle peut être appliquée au sein d'une formule.
  \begin{align*}
    \Diamond \phi & \longrightarrow \neg \Box \neg \phi \\
    \exists x : \phi & \longrightarrow \neg \forall x : \neg \phi \\
    \neg \neg \phi & \longrightarrow \phi \\
    \phi_1 \Leftrightarrow \phi_2 & \longrightarrow (\phi_1 \Rightarrow \phi_2) \land (\phi_2 \Rightarrow \phi_1) \\
    \phi_1 \Rightarrow \phi_2 & \longrightarrow \neg \phi_1 \lor \phi_2 \\
    \phi_1 \land \phi_2 & \longrightarrow \neg(\neg \phi_1 \lor \neg \phi_2) \\
    \TRUE & \longrightarrow \neg \FALSE \\
    \Box \Box \phi & \longrightarrow \Box \phi \\
  \end{align*}
\end{defin}

\paragraph{Remarques}

On pourrait essayer ensuite d'étendre ce système de réécriture, par exemple en développant les formules du type $A \land (B \lor C) \longrightarrow (A \land C) \lor (A \land B)$ en s'inspirant du calcul d'une forme clausale pour une formule FOL.
Cela pourrait permettre d'identifier plus de formules vers une même formule TLA$^+$ mais risque d'impliquer un temps exponentiel lors du calcul.


\paragraph{Propriétés}

On montre différentes propriétés vérifiées par ce système de réécriture

\begin{prop} \emph{Correction du système de réécriture} \\
  Pour $\psi_i, \psi_f$ des formules TLA$^+$ telles que $\psi_i \longrightarrow^* \psi_f$,
  la formule FOML $\psi_i \Leftrightarrow \psi_f$ est valide.
\end{prop}

On peut démontrer cette propriété par récurrence sur la longueur de la dérivation.
Le cas où $\psi_i = \psi_f$ est immédiat. Il suffit de vérifier que la propriété est vraie dans le cas où $\psi_i \longrightarrow \psi_f$.
On remarque que dans chaque étape de réécriture les formules à gauche et à droite sont équivalentes.
L'équivalence est préservée en insérant ces formules dans un contexte (càd une formule plus grande avec un emplacement libre), ce qui conclut la démonstration.

\begin{prop} \emph{Terminaison du système de réécriture} \\
  Pour $\psi$ une formule TLA$^+$ il n'existe pas de dérivation infinie partant de $\psi$.
\end{prop}

On peut le montrer en utilisant un ordre bien fondé sur les formules TLA$^+$ tel que pour chaque étape de réécriture la formule finale soit strictement inférieure à la formule initiale.
On considère l'ordre lexicographique suivant :
pour $\phi$ une formule FOML, on pose
\begin{comment}
\begin{itemize}
\item $a$ est la hauteur maximale d'un symbole $\Leftrightarrow$ dans $\phi$ (on peut représenter $\phi$ sous forme d'un arbre où un noeud est un opérateur et ses fils sont ses arguments)
\end{itemize}
\end{comment}

\end{document}