\documentclass[12pt]{article}

\usepackage[utf8]{inputenc}
\usepackage[french]{babel}
\usepackage{amsmath, amssymb}
\usepackage{stmaryrd}


\title{Coalescing - Brouillon}
\author{Raphaël Le Bihan}

\newcommand{\defeq}{\ensuremath{\; \hat{=} \;}}
\newcommand{\FOL}{\ensuremath{\textup{\tiny{}FOL}}}
\newcommand{\false}{\textup{ff}}
\newcommand{\true}{\textup{tt}}
\newcommand{\M}{\ensuremath{\mathcal{M}}}
\newcommand{\I}{\ensuremath{\mathcal{I}}}
\newcommand{\dom}{\ensuremath{\textup{ dom }}}

\usepackage{xcolor}
\newcommand{\raph}[1]{\textcolor{red}{#1}}

\begin{document}

\maketitle

\paragraph{Contre-exemple}

Opérateur : \[ d(x) \triangleq \nabla x \]
Formules :
\[ \phi = \forall z : \nabla z \land d(\nabla z) \]
\[ \widetilde{\phi} = \forall z : \nabla z \land \nabla \nabla z \]
Formules FOL :
\[ \phi_\FOL = \forall z : \framebox{$\lambda z . \nabla z$}(z) \land \framebox{$d_{\nabla z}$}(\framebox{$\lambda z . \nabla z$}(z)) \]
\[ \widetilde{\phi}_\FOL = \forall z :  \framebox{$\lambda z . \nabla z$}(z) \land \framebox{$\lambda z . \nabla \nabla z$}(z) \]
Modèle FOL \M :
\[ \dom \M = \{ \true, \false \} \]
\[ \I(\framebox{$\lambda z . \nabla z$}) :
  \left \{
    \begin{array}{r c l}
      \true & \mapsto & \true \\
      \false & \mapsto & \false
    \end{array}
  \right . \]
\[ \I(\framebox{$\lambda z . \nabla \nabla z$}) :
  \left \{
    \begin{array}{r c l}
      \true & \mapsto & \false \\
      \false & \mapsto & \true
    \end{array}
  \right . \]
Complétion de $\M$ en $\M_d$ :
\[ \I_d(\framebox{$d_\epsilon$}) : a \mapsto
  \llbracket \framebox{$\lambda x . \nabla x$}(x) \rrbracket_{\M[x \mapsto a]} \]
\[ \I_d(\framebox{$\lambda z . \nabla z$}) = \I(\framebox{$\lambda z . \nabla z$}) \]
\[ \I_d(\framebox{$\lambda z . \nabla \nabla z$}) = \I(\framebox{$\lambda z . \nabla \nabla z$}) \]

% La preuve par récurrence échoue alors car il est faux que pour toute sous formule $\psi$ de $\phi$ et $y$ liste de variables libres, $\llbracket \psi_\FOL^y \rrbracket_{\M_d} = \llbracket \widetilde{\psi}_\FOL^y \rrbracket_{\M_d}$.
% En effet :
Alors :

\[ \llbracket \framebox{$d_{\nabla z}$}(\framebox{$\lambda z . \nabla z$}(z)) \rrbracket_{\M_d}
  \neq
  \llbracket \framebox{$\lambda z . \nabla \nabla z$}(z) \rrbracket_{\M_d} \]
Car :
\begin{align*}
  \llbracket \framebox{$d_{\nabla z}$}(\framebox{$\lambda z . \nabla z$}(z)) \rrbracket_{\M_d}
  % = \I_d(\framebox{$d_\epsilon$})(\llbracket \framebox{$\lambda z . \nabla z$}(z)) \rrbracket_{\M_d}) \\
  &= \llbracket \framebox{$\lambda x . \nabla x$}(x) \rrbracket_{\M_d[x \mapsto \llbracket \framebox{${}_{\lambda z . \nabla z}$}(z)) \rrbracket_{\M_d}} \\
  &= \llbracket x \rrbracket_{\M_d[x \mapsto \llbracket \framebox{${}_{\lambda z . \nabla z}$}(z)) \rrbracket_{\M_d}} \\
  &= \llbracket \framebox{${\lambda z . \nabla z}$}(z)) \rrbracket_{\M_d} \\
  &= \llbracket z \rrbracket_{\M_d}
\end{align*}
Mais :
\begin{align*}
  \llbracket \framebox{$\lambda z . \nabla \nabla z$}(z) \rrbracket_{\M_d}
  &= \left \{
    \begin{array}{l l}
      \true & \textup{ si } \llbracket z \rrbracket_{\M_d} = \false \\
      \false & \textup{ sinon }
    \end{array}
               \right . \\
  &\neq \llbracket z \rrbracket_{\M_d}               
\end{align*}

\raph{%
  Cela veut dire qu'il faut trouver une autre interprétation des \framebox{$d_{\vec{\epsilon}}$} dans $\M_d$, pour pouvoir satisfaire la propriété voulue.}
\end{document}